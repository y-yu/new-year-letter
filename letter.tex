\documentclass{jletteraddress}

\usepackage{amsmath}
\usepackage[papersize={100mm, 148mm}, margin=0mm]{geometry}

\newcommand{\bra}[1]{\mathinner{\left\langle{#1}\right|}}
\newcommand{\ket}[1]{\mathinner{\left|{#1}\right\rangle}}
\newcommand{\braket}[2]{\mathinner{\left\langle{#1}\middle|#2\right\rangle}}

\sendername{吉村 優}
\senderaddressa{茨城県つくば市天王台1-1-1}
\senderaddressb{筑波大学第三エリア3C212}
\senderpostcode{3058573}

\renewcommand{\baselinestretch}{1.2}

\begin{document}

\pagestyle{empty}

\addaddress
  {山田 太郎}{様}
  {1138654}
  {東京都文京区本郷7-3-1}
  {}

\newpage
\newgeometry{
  top=10mm,
  left=10mm,
  right=10mm,
  bottom=5mm
}

\begin{abstract}
  \footnotesize
  この文章では、新年の挨拶を述べるための手紙である``年賀状''を作る方法について述べる。
  従来の年賀状を作成するソフトウェアは数式やソースコードの埋め込みが貧弱であったが、
  我々はそれを解決す方法として組版ソフト\LaTeX を利用する方法を示す。

  In this paper, we describe how to make a letter to hello new year.
  For conventional softwares that makes those letter,
  it's difficult to write mathmatical expressions or
  programming source code.
  We show how we solve that problem with \LaTeX.
\end{abstract}

\section*{Introduction}

年賀状は年の始めに互いに送信する手紙のことである。
近年はLINEなどに対抗するため、年賀状に次のような複雑な数式を埋め込みたい
ニーズが存在する。

{\scriptsize
\begin{align*}
  \left|\braket{+}{\varphi_{0 \oplus b, 0 \oplus b}}\right|^2 &= \left\{
\begin{array}{l}
\left|\braket{+}{\alpha\ket{0} + \beta\ket{1}}\right|^2 = \left|\frac{1}{\sqrt{2}}(\alpha + \beta)\right|^2 \\
\left|\braket{+}{\beta\ket{0} + \alpha\ket{1}}\right|^2 = \left|\frac{1}{\sqrt{2}}(\beta + \alpha)\right|^2
\end{array}
\right\} \\
&= \frac{(\alpha + \beta)^2}{2}
\end{align*}
}

\noindent
この文章ではこのような複雑な数式を埋め込むために\LaTeX を
利用した解決方法について述べる。

\begin{flushright}
  \tiny This letter was generated by \LaTeX.
\end{flushright}

\end{document}